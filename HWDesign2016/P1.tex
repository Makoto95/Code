\documentclass[10pt,a4j,dvipdfmx]{jsarticle}
\usepackage[utf8]{inputenc}
\usepackage[dvipdfmx]{graphicx}
\usepackage[usenames,dvipdfmx]{color}
\usepackage{amsmath}
\usepackage{bm}
\usepackage[left=19.05mm, right=19.05mm, top=25.40mm, bottom=25.40mm]{geometry}
\usepackage{tikz}
\usepackage{circuitikz}
\usepackage{siunitx}
\usepackage{listings}
\usepackage{float}
\usepackage{hyperref}

\lstset{%
  language={C},
  basicstyle={\small},%
  identifierstyle={\small},%
  commentstyle={\small\itshape},%
  keywordstyle={\small\bfseries},%
  ndkeywordstyle={\small},%
  stringstyle={\small\ttfamily},
  frame={tb},
  breaklines=true,
  columns=[l]{fullflexible},%
  numbers=left,%
  xrightmargin=0zw,%
  xleftmargin=3zw,%
  numberstyle={\scriptsize},%
  stepnumber=1,
  numbersep=1zw,%
  lineskip=-0.5ex%
}

\usepackage{fouriernc}
\usepackage[scaled]{helvet}
\usepackage[T1]{fontenc}
\renewcommand{\ttdefault}{fvm}

\let\oldthefootnote\thefootnote
\def\thefootnote{{\color{Magenta}\oldthefootnote}}

\newcommand{\enhance}[1]{{\gtfamily\sffamily#1}}
\makeatletter
\def\@jikkenname{}
\def\@jikkennum{}
\def\@reportname{}
\def\@studentnumber{}
\def\@studentname{}
\def\@studentdepartment{}
\def\@friendnames{}
\def\@groupnumber{}
\newcommand{\jikkenset}[2]{\def\@jikkennum{#1}\def\@jikkenname{#2}}
\newcommand{\studentset}[3]{\def\@studentnumber{#1}\def\@studentname{#2}\def\@studentdepartment{#3}}
\newcommand{\reportnameset}[1]{\def\@reportname{#1}}
\newcommand{\friendname}[1]{\def\@friendnames{#1}}
\newcommand{\groupnumber}[1]{\def\@groupnumber{#1}}
\renewcommand{\maketitle}{
\noindent{\color{RoyalPurple}\hrule height 1pt \hfill}
\vspace{5pt}
\begin{center}
\enhance{{\Large{電気電子情報第一(前期)実験}}}\\[7pt]
\enhance{{\Huge\textbf{\@jikkennum{}. \@jikkenname}}}\\[5pt]
\enhance{{\LARGE{\@reportname}}}\\[15pt]
\@studentnumber\ \ \ \@studentname{}(\@studentdepartment{})\\[1pt]
共同実験者: \@friendnames(第\@groupnumber{}班)\\[1pt]
\today
\end{center}
\vspace{-10pt}
\noindent{\color{RoyalPurple}\hrule height 1pt \hfill}
}
\makeatother
\jikkenset{P1}{電気回路の基礎}
\reportnameset{考察レポート}
\studentset{03-160441}{土屋潤一郎}{工学部 電子情報工学科}
\friendname{井上友貴、田中大幹、坂口達彦}
\groupnumber{28}

\begin{document}
\maketitle

\section{実験の概要}
\begin{description}
 \item[第1日]計測機器の使用習熟およびはんだによる基板実装習熟。RLC直列回路の周波数特性及びステップ応答の計測。
 \item[第2日]RC四端子網回路の周波数特性及びステップ応答の計測。
\end{description}
\section{考察}
\subsection{伝達関数の極・零点を用いた周波数特性の説明}
s領域(平面)での伝達関数$H(s)$を

\begin{equation}
H(s) = H\bullet\frac{(s-s_{z1})(s-s_{z2})(s-s_{z3})...}{(s-s_{p1})(s-s_{p2})(s-s_{p3})...}
\end{equation}

とすると、その極と零点を用いて、振幅特性$|H(s)|$と位相特性$\b{Arg}$$|H(s)|$が以下のように表せる。

\begin{eqnarray}
\left|H\left(s\right)\right| &=&
\left|H\right|
\frac{\prod_{i} d_{zi}\left(s\right)}
{\prod_{i}d_{pi}\left(s\right)}\\
\b{Arg}\left|H\left(s\right)\right| &=& \sum_{i}\theta_{zi}\left(s\right) - \sum_{i}\theta_{pi}\left(s\right) \\
\end{eqnarray}
ただし、$d_{zi}$はi番目の零点とsとの距離、$d_{pi}$はi番目の極とsとの距離、
$\theta_{zi}$はi番目の零点からsへの偏角、$\theta_{pi}$はi番目の極からsへの偏角である。

\subsubsection{RLC直列共振回路}
2端子回路なので、出力を時間に対する電流波形と考えると、
その伝達関数は、

\begin{equation}
H(s) = \frac{I}{V} = Y
\end{equation}

と、アドミタンスに等しい。

今回28班が用いた回路を簡素化し(コイル内部以外の抵抗成分をまとめる)、
コイル素子を等価回路に置き換えると、
図1のようになる。

\begin{figure}[H]
  \centering
  \includegraphics[width=8cm]{RLCkettei.png}
  \caption{測定に用いたRLC直列回路の等価回路}
\end{figure}

この合成アドミタンスを用いて伝達関数を整理すると、

\begin{equation}
H(s) = \frac{1}{R} \bullet \frac{s^{3}+\frac{R_{L}}{L}s^{2}+\frac{1}{C_{L}L}s}
{s^{3}
+\left(\frac{R_{L}}{L}+\frac{1}{C_{L}R}+\frac{1}{CR}\right)s^{2}
+\left(\frac{1}{C_{L}L}+\frac{R_{L}}{C_{L}LR}+\frac{R_{L}}{CLR}\right)s
+\frac{1}{CC_{L}LR}
}
\end{equation}

これに、
\begin{eqnarray}
R &=& 39.0\left[\si{\ohm}\right] \\
R_{s} &=& 5.5\left[\si{\ohm}\right] \\
L_{s} &=& 45.6\bullet10^{-6}\left[\si{\henry}\right] \\
C_{p} &=& 550\bullet10^{-12}\left[\si{\farad}\right] \\
C &=& 470\bullet10^{-12}\left[\si{\farad}\right] \\
\end{eqnarray}
を代入して、極と零点を求める。

伝達関数の極と零点は3つずつで、図2のような配置となる。

値としては、零点が
\begin{eqnarray}
s &=& 0 \\
s &=& -6.05×10^4-6.32×10^6 j \\
s &=& -6.05×10^4+6.32×10^6 j
\end{eqnarray}
で、極が
\begin{eqnarray}
s &=& -9.85×10^7 \\
s &=& -1.54×10^5-4.64×10^6 j \\
s &=& -1.54×10^5+4.64×10^6 j
\end{eqnarray}
である。

\begin{figure}[H]
  \centering
  \includegraphics[width=5cm]{token.png}
  \caption{RLC直列回路の周波数特性測定に用いた回路のアドミタンスの極と零点}
\end{figure}

さて、周波数はこの虚軸上正の部分を動くから、まず原点では1つの零点との距離が0なので振幅の利得が0なのは当然だが、極・零点ともに3つずつあるのでその振る舞いは一見して共振周波数がわかるほど自明ではない。位相についても同様である。
従って、これらの周波数特性を、(2)式及び(3)式に基づいて数値計算して(実験結果のデータセットとともに)プロットするプログラムを用意した(レポート末付録資料、参考コード1)。その結果のグラフが、図3である。

\begin{figure}[H]
  \centering
  \includegraphics[width=16cm]{P1TF.png}
  \caption{周波数特性の実測値と理論値}
\end{figure}

共振周波数のずれを無視すれば、グラフの概形は一致しているから、伝達関数(6)式の極と零点の位置に依って概形を説明できたと言って良いだろう。
一方、共振周波数の数値については説明できない。原因としては、コイルの並列キャパシタンス成分が挙げられる。
LCRメーターで測定したコイルの並列キャパシタンスは550p[F]であったが、基板にはんだ付けしたことによってこの値が変化した可能性がある。

再び数値計算とプロットを用いる。
レポート末付録資料の参考コード2によって描いた、コイルの並列キャパシタンスが24p[\si{\farad}]だと仮定した場合の振幅特性の理論値曲線が図4である。

\begin{figure}[H]
  \centering
  \includegraphics[width=16cm]{P1analyze.png}
  \caption{周波数特性の実測値と理論値($Cp=24*10^{-12}\left[\si{\farad}\right]$の場合)}
\end{figure}

このように、コイルを基板にはんだ付けしたことによってコイルの並列キャパシタンスが低下したと考えられる。
一方、参考資料[2]p17にはインダクタをプリント配線板に実装すると逆に並列キャパシタンスが増加(して自己共振周波数が低下する)旨の記載がある。
従って、LCRメーターでの計測に誤りがあった可能性もある。
いずれにせよ、回路上ではこの並列キャパシタンスの値が小さかったと考えるべきだろう。

一方、アドミタンスのピークやのQ値のずれは、接触抵抗によって説明できる。全体の直列抵抗値が$10^{1}$のオーダーなので、ひと桁オームの接触抵抗でも十分アドミタンスのピークやのQ値が大きく変わりうる。

\subsubsection{LPF}
図5に、測定に用いたRC低域通過フィルタを、素子の値とともに示す。
\begin{figure}[H]
  \centering
  \includegraphics[width=8cm]{LPF.png}
  \caption{測定に用いたLPF}
\end{figure}

この4端子回路の基本行列は、
\begin{equation}
F =
\left[
\begin{array}{rr}
1 & R \\
0 & 1 \\
\end{array}
\right]
\left[
\begin{array}{rr}
1 & 0 \\
sC & 1 \\
\end{array}
\right]
=
\left[
\begin{array}{rr}
1+sRC & R \\
sC & 1 \\
\end{array}
\right]
\end{equation}

伝達関数$\frac{V_{out}\left(s\right)}{V_{in}\left(s\right)}$は、1行1列目の成分の逆数だから、
\begin{equation}
H\left(s\right) = \frac{1}{1+sRC}
\end{equation}
従って、極が実数軸上に一つである。(図)
抵抗値と容量値を代入してその値を求めると、$s = -10^4$である。
\begin{figure}[H]
  \centering
  \includegraphics[width=5cm]{token.png}
  \caption{周波数特性測定に用いたLPFの入出力電圧比の極}
\end{figure}

周波数は虚軸上正の部分を動くから、
振幅特性は周波数0で最大、周波数が増大するにつれ(2)式の分母が大きくなって減少し、
位相特性は周波数0で0、周波数が増大するにつれ(3)式の第二項が大きくなって減少する。

\subsubsection{HPF}
図5に、測定に用いたRC低域通過フィルタを、素子の値とともに示す。
\begin{figure}[H]
  \centering
  \includegraphics[width=8cm]{HPF.pdf}
  \caption{測定に用いたHPF}
\end{figure}

この4端子回路の基本行列は、
\begin{equation}
F =
\left[
\begin{array}{rr}
1 & \frac{1}{sC} \\
0 & 1 \\
\end{array}
\right]
\left[
\begin{array}{rr}
1 & 0 \\
\frac{1}{R} & 1 \\
\end{array}
\right]
=
\left[
\begin{array}{rr}
1+\frac{1}{sCR} & \frac{1}{sC} \\
\frac{1}{R} & 1 \\
\end{array}
\right]
\end{equation}

LPFと同様に、伝達関数は、
\begin{equation}
H\left(s\right) = \frac{1}{1+\frac{1}{sCR}} = \frac{CR}{CR+\frac{1}{s}}
\end{equation}
従って、極が実数軸上に一つ、原点が零点である。
抵抗値と容量値を代入して極の値を求めると、$s = -10^4$である(図6)。
\begin{figure}[H]
  \centering
  \includegraphics[width=5cm]{token.png}
  \caption{周波数特性測定に用いたHPFの入出力電圧比の極と零点}
\end{figure}

周波数は虚軸上正の部分を動くから、
振幅特性は周波数0で0、周波数の増大による極との距離と零点との距離の比が1に近づいていく。
位相特性は(3)式の第1項が常に$90^{\circ}$で、周波数が増大するにつれ第2項が0から大きくなって減少する。

\subsubsection{APF}
一次のAPFは伝達関数が
\begin{equation}
H\left(s\right) = H \frac{s-\alpha}{s+\alpha}
\end{equation}
である(図9)から、
虚数軸上を周波数が動くと、
周波数0では(3)式の第1項が$180^{\circ}$、第2項が$0^{\circ}$、
周波数が増大するにつれその差が縮まり、極限ではどちらも$90^{\circ}$に収束していく。

\begin{figure}[H]
  \centering
  \includegraphics[width=5cm]{token.png}
  \caption{周波数特性測定に用いたAPFの入出力電圧比の極と零点}
\end{figure}

\subsection{インパルス応答を用いた周波数特性の説明}
時間領域でのインパルス応答は周波数領域では伝達関数そのものであり、入力と伝達関数の積が出力である関係は時間領域では畳込み積分となる。
従ってまず、2.1で用いた伝達関数を時間領域に戻してインパルス応答を求める。

\subsubsection{RLC直列共振回路}
\begin{equation}
H(s) = \frac{1}{R} \bullet \frac{s^{3}+as^{2}+bs}
{s^{3}
+cs^{2}
+ds
+e}
}
\end{equation}

\subsubsection{LPF}
\subsubsection{HPF}
\subsubsection{APF}


\clearpage
\section{付録資料}
\lstinputlisting[caption=参考コード1]{P1TFunctionSuikouzumi.py}
\lstinputlisting[caption=参考コード2]{P1analyze.py}

\section{参考資料}
[2]\\
コイルを使う人のための話 (サガミ エレク株式会社 技術統括部) \\
\href{http://www.sagami-elec.co.jp/file/tech/coil_doc_100j.pdf}{http://www.sagami-elec.co.jp/file/tech/coil_doc_100j.pdf}\\
\end{document}
